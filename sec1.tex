結晶質岩中のマイクロクラックは、岩盤の長期的な物質輸送特性や力学的な挙動に影響を与えると
考えられる。マイクロクラックの多くは肉眼で直接観測することは難しく、薄片を作成し偏光顕微鏡で
観察するなどして調べる必要がある。このような検査は非常に労力を要する上、
検査対象となる岩の状態をその場で非破壊的に調べる際には利用できない。
そのため、その場かつ非破壊的にマイクロクラックの状態を調べる技術があれば、
岩盤の現在の状態や長期の将来的な挙動を予想する上で有用な情報を提供刷ることができる。
そのような観点のもと、著者らは、超音波を使った花崗岩コア供試体の音響的性質の評価に
取り組んで来た。その結果としてこれまでに、粗大な結晶粒を有する花崗岩であっても、
数cm程度の距離であれば、数mmスケールの超音波を透過させることができることを示した。
また、レーザードップラー振動計を用いて波動場を詳細に観測することで、
鉱物粒と弾性波の相互作用を可視化できること、試料の部位や状態に応じて表面波速度に
有意な変化が現れることも明らかにしてきた。
これら音速の変化は、波動場の可視化結果からマイクロクラックの状態を
反映したものと予想されるものの、き裂の存在によって岩石の音響的性質が変化することの
より確実な証拠を見出すことが課題として残されていた。
花崗岩では、造岩鉱物自体が異方性を持ち、き裂が存在せずとも、異方性や散乱による
弾性波の減衰が起きる。そのため、みかけの異方性や減衰が観測されることから、
き裂の影響であるとそのまま結論を導くことはできない。
一方で、き裂は、岩盤が受ける応力場に起因して発生や進展した場合、応力場に応じた
配向性を示すことや、特定の鉱物に集中して発生する可能性が予想される。
従って、き裂の発生や成長に伴う音響異方性の変化や、特定の鉱物種の周辺で
生じる特異な散乱、鉱物種の含有割合におうじた音響的性質の変化を観察することが
き裂と弾性波の相互作用を理解し、計測波形からき裂の状態を推定する上で必要となる。
そのためには、鉱物粒からコアサンプルのスケールまで種々の空間解像度で弾性波散乱や
音響異方性を調べる方法が必要となる。
従来、岩石コアの音響的性質の評価では、圧電トランスデューサでコアサンプルを
透過した波動を計測しその音速や異方性を調べることが行われてきた。
この方法では、コア全体の平均的な性質は調べることができるものの、コアの部位や
鉱物の分布状況に応じてた音響的な性質の評価は行うことができない。
また、超音波を透過させることのできる方向は供試体の形状によって決まるため、
状況によっては、透過波を計測ができない方向もあり、音響異方性を調べることができない
場合もある。このことから、クラックの状態を知るための弾性波計測は、
供試体表面の任意の位置と方向で実施可能な方法を用いることが望ましい。
本年度の研究では、そのような計測の実現に向け、同一表面において、任意の方向と伝播
距離で超音波の送信と受信と行う方法を提案し、音響異方性の評価に利用可能である
ことを示した。具体的には、コアサンプル中央の
20mm角の領域が示す音響異方性を超音波波形から推定し、マイクロクラックに起因した
直交異方性が観察できることを明らかにする。
以下では、新たに構築した超音波計測系と 実験に用いた花崗岩コア供試体について述べる。
特に、強い超音波を送信するために新たに導入した集束型接触探触子の仕様と
導入の意図を述べる。計測点の配置や波形収録に関する設定について示した後、
実際に構築した計測系で得た超音波波形を示す。
次に、超音波波形から岩石コア試料の音響的性質を得るために、行った
波形処理の方法とその結果得られた音速やピーク周波数を示す。
最後に、入射方向によるそれら音響的性質の変化を示し。花崗岩よく知られている通り、
直交異方性を示すことが、表面波計測結果からも示されること、さらに、
波形振幅と音速の相関から、異方性はき裂に起因したものであることが示唆されることを
述べ、まとめと今後の課題について述べる。

 結晶質岩中のマイクロクラックは,岩盤の物質輸送特性や力学的挙動とその長期的な変化に
影響を与えると考えられる.しかしながら,マイクロクラックの多くは肉眼で直接観測することが
難しく,薄片を作成して偏光顕微鏡下で観察する等の方法によって検出や定量化をする必要
がある.このような検査は多大な労力を要する上,岩石の状態をその場で非破壊的に調べる
目的には利用できない.そのため,岩石に含まれるマイクロクラックの状態をその場で非破壊的
に調べる技術があれば,岩盤の現在の状態を知り,将来的な挙動を予測をする上で有用な情報
を提供することができると考えられる.

 このような観点のもと,著者らは,超音波を使った花崗岩コア供試体の音響的性質の評価に
取り組んで来た.その結果と,粗大な結晶粒を有する花崗岩であっても,コアサンプルスケール
の距離であれば,1MHz程度の周波数帯域数をもつ超音波を透過させることができることを
示してきた.また,レーザードップラー振動計を用いて,岩石表面の超音波伝播挙動を詳細に
観測することで,鉱物粒と弾性波の相互作用を可視化できることや,試料の部位や方向に応じて
表面波速度に有意な変化が現れることも明らかにしてきた.このような,高周波の弾性波が示す
音響的性質と,岩石物性やマイクロクラックとの相関について理解を深めれば,弾性波を使った
岩石物性やき裂の評価が可能になると期待される.実際,これまでに行った超音波伝播挙動の
可視化結果から判断し,音速のばらつきや異方性は,マイクロクラックの状態を反映したもので
あることが示唆されている.しかしながら,花崗岩に代表される結晶質岩では,造岩鉱物自体が
異方性を持ち,き裂が存在せずとも,音響異方性や散乱減衰が生じるため,みかけの異方性や
減衰が観測されることをもって,き裂の影響があるとの結論を導くことはできない.このことから,
き裂の存在によって岩石の音響的性質が変化することのより確実かつ直接的な証拠を見出す
ことが,解決すべき課題として残されていた.

 き裂は,岩盤が受ける応力場に起因して発生や進展した場合,応力場に準じた配向性を
示すと考えられる.また,熱膨張によって生じたき裂は,特定の鉱物に集中して発生する
ことも予想される.そのため,結晶主軸の配向と一致しない音響異方性や,特定の鉱物種周辺で
生じる特異な散乱を観察し,弾性波とき裂の相互作用モデルを構築することができれば,
マイクロクラックの超音波非破壊評価につなげることができる.ただし,そのためには,鉱物粒の
スケールからコアサンプルサイズのスケールまで,種々の空間解像度で弾性波の散乱や
音響異方性を調べることが計測技術面での課題となる.

 従来,岩石コアの音響的性質の評価では,コアサンプルを透過した波動を圧電トランスデューサ
で計測し,その音速や異方性を調べることが行われてきた.この方法では,コア全体の平均的な
性質は調べることができるものの,コアの特定部位において,鉱物の分布状況によって音響的性質
がどのように変化するかといった評価までをを行うことができない.また,超音波を透過させることの
できる方向は供試体形状によって決まるため,状況によっては透過波計測ができない方向もあり,
特に局所的な音響異方性を調べることは非常に困難である.一方,マイクロクラック評価を
目的とした弾性波計測では,供試体表面の任意の位置と方向で実施可能であることが必要とされる.

以上のことを踏まえ,本年度の研究では,岩石コア供試体表面において,任意の方向と任意の
伝播距離で超音波の送信と受信と行う方法を提案し,局所的な音響異方性の評価に利用する
ことを試みた.具体的には,強い超音波を励起するための線集束型トランスデューサを設計,
制作して計測に用い,試料表面近傍を伝播する平面波を計測する.この方法で,コアサンプル
中央の20mm角の領域が示す音響異方性を,超音波波形から推定し,マイクロクラックに起因
した直交異方性が観察できることを示す.

本報告,以下では,はじめに新たに構築した超音波計測系と,実験に用いた花崗岩コア
供試体について述べる.特に,強い超音波を送信するために今年度から導入した,線集束型
接触探触子の仕様と導入の意図を述べる.続いて,計測点の配置や波形収録に関する設定
について述べた後,本年度の計測で得られた超音波波形を示す.最後に,計測した超音波波形
から岩石コア試料の音響的性質を評価することを目的として行った一連の波形処理とその
結果を示す.最後に,入射方向による音響的性質の変化を示し,花崗岩試料が直交異方性を
示すことを,表面波計測によって観察できること,さらに,波形振幅と音速の相関から,異方性
の起源はき裂であると考えられることを示した後,まとめと今後の課題について述べる.
